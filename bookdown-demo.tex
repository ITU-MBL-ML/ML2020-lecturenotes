\documentclass[]{book}
\usepackage{lmodern}
\usepackage{amssymb,amsmath}
\usepackage{ifxetex,ifluatex}
\usepackage{fixltx2e} % provides \textsubscript
\ifnum 0\ifxetex 1\fi\ifluatex 1\fi=0 % if pdftex
  \usepackage[T1]{fontenc}
  \usepackage[utf8]{inputenc}
\else % if luatex or xelatex
  \ifxetex
    \usepackage{mathspec}
  \else
    \usepackage{fontspec}
  \fi
  \defaultfontfeatures{Ligatures=TeX,Scale=MatchLowercase}
\fi
% use upquote if available, for straight quotes in verbatim environments
\IfFileExists{upquote.sty}{\usepackage{upquote}}{}
% use microtype if available
\IfFileExists{microtype.sty}{%
\usepackage[]{microtype}
\UseMicrotypeSet[protrusion]{basicmath} % disable protrusion for tt fonts
}{}
\PassOptionsToPackage{hyphens}{url} % url is loaded by hyperref
\usepackage[unicode=true]{hyperref}
\hypersetup{
            pdftitle={MBL549 Machine Learning in Architecture},
            pdfauthor={Özgün Balaban},
            pdfborder={0 0 0},
            breaklinks=true}
\urlstyle{same}  % don't use monospace font for urls
\usepackage{natbib}
\bibliographystyle{apalike}
\usepackage{longtable,booktabs}
% Fix footnotes in tables (requires footnote package)
\IfFileExists{footnote.sty}{\usepackage{footnote}\makesavenoteenv{long table}}{}
\usepackage{graphicx,grffile}
\makeatletter
\def\maxwidth{\ifdim\Gin@nat@width>\linewidth\linewidth\else\Gin@nat@width\fi}
\def\maxheight{\ifdim\Gin@nat@height>\textheight\textheight\else\Gin@nat@height\fi}
\makeatother
% Scale images if necessary, so that they will not overflow the page
% margins by default, and it is still possible to overwrite the defaults
% using explicit options in \includegraphics[width, height, ...]{}
\setkeys{Gin}{width=\maxwidth,height=\maxheight,keepaspectratio}
\IfFileExists{parskip.sty}{%
\usepackage{parskip}
}{% else
\setlength{\parindent}{0pt}
\setlength{\parskip}{6pt plus 2pt minus 1pt}
}
\setlength{\emergencystretch}{3em}  % prevent overfull lines
\providecommand{\tightlist}{%
  \setlength{\itemsep}{0pt}\setlength{\parskip}{0pt}}
\setcounter{secnumdepth}{5}
% Redefines (sub)paragraphs to behave more like sections
\ifx\paragraph\undefined\else
\let\oldparagraph\paragraph
\renewcommand{\paragraph}[1]{\oldparagraph{#1}\mbox{}}
\fi
\ifx\subparagraph\undefined\else
\let\oldsubparagraph\subparagraph
\renewcommand{\subparagraph}[1]{\oldsubparagraph{#1}\mbox{}}
\fi

% set default figure placement to htbp
\makeatletter
\def\fps@figure{htbp}
\makeatother

\usepackage{booktabs}
\usepackage{amsthm}
\makeatletter
\def\thm@space@setup{%
  \thm@preskip=8pt plus 2pt minus 4pt
  \thm@postskip=\thm@preskip
}
\makeatother

\title{MBL549 Machine Learning in Architecture}
\author{Özgün Balaban}
\date{}

\begin{document}
\maketitle

{
\setcounter{tocdepth}{1}
\tableofcontents
}
\chapter{Syllabus}\label{intro}

\section{Class Schedule}\label{class-schedule}

\textbf{Spring 2020} Tuesday 13:30-1630

\section{Content}\label{content}

The course is an introduction to Machine Learning methods with examples
from architectural design and creative coding. The course follows a
hands on approach with many examples that will be developed during the
course. The course includes topics from machine learning such as linear
regression, unsupervised learning, supervised learning, reinforcement
learning and finally neural networks. \#\# Aims To have students acquire
practical knowledge on the tools of machine learning and different
methodologies. To have students knowledge of applying these techniques
to creative design process.

\section{Conduct}\label{conduct}

The first half of the term includes introduction to the various machine
learning topics. In this phase there will be practical coding sessions
and some assignments which will be graded. In the second half of the
term we will focus on a group project which will be developed using
neural networks.

\section{Assessment}\label{assessment}

The grading for the course is as follows: hands-on practices, 50\%;
final group project 50\%.

\subsection{Assignments}\label{assignments}

TBA

\subsection{Final Project}\label{final-project}

TBA

\section{Software}\label{software}

We will use Python in an Anaconda environment.

\section{Outline}\label{outline}

\textbf{Week 1} Introduction

\textbf{Week 2} Refresher on Python

\textbf{Week 3} Data scraping \& Linear Regression

\textbf{week 4} Unsupervised Learning

\textbf{Week 5} Supervised Learning

\textbf{Week 6} Reinforcement Learning

\textbf{Week 7} Introduction to Neural Networks

\textbf{Week 8} Break

\textbf{Week 9} Neural Network examples - Project Discussion

\textbf{Week 10} Neural Network examples

\textbf{Week 11} Neural Network examples

\textbf{Week 12} Projects

\textbf{Week 13} Projects

\textbf{Week 14} Projects

\textbf{Week 15} Projects

\chapter{Week 1}\label{week-1}

In the first week we had a talk about the course syllabus and discussed
about Machine Learning examples.

For the second week we will do a refresher on Python. Please bring your
laptops and install Anaconda with Python 3.7.

\section{Install Anaconda}\label{install-anaconda}

\begin{enumerate}
\def\labelenumi{\arabic{enumi}.}
\tightlist
\item
  Install Anaconda\\
  \url{https://www.anaconda.com/distribution/\#download-section}
\item
  Click on start menu and choose anaconda prompt'(windows) OR run
  terminal (osx/ubuntu)
\item
  Create a virtual environment (e.g. `mbl-ml', could be any name you
  like):\\
  • \textgreater{} conda create -n mbl-ml python=3.7 ipykernel
\item
  Activate the virtual environment:\\
  • \textgreater{} source activate mbl-ml
\item
  Install dependencies:\\
  • Change directory to where the given file (requirements.txt)
  resides.\\
  • \textgreater{} cd (just drag your folder into the cmd/terminal)\\
  • \textgreater{} pip install -r requirements.txt\\
\item
  Make kernel accessible in jupyter notebook:\\
  • \textgreater{} python -m ipykernel install --user --name mbl-ml\\
\item
  Go to the directory that you would like to save your notebooks •
  \textgreater{} cd (just drag your folder into the cmd/terminal)
\item
  Run jupyter notebook: • \textgreater{} jupyter notebook
\item
  Create new file and set kernel to `mbl-ml'
\item
  To shut down jupyter: • Press cmd `C' or \textgreater{} ctrl `C' at
  the terminal/cmd prompt
\item
  To deactivate virtual environment: • \textgreater{} conda deactivate
\end{enumerate}

\bibliography{book.bib,packages.bib}

\end{document}
